\chapter{Background and related Work}
\label{ch:background}
(TODO) Anpassen zu neuem content

This chapter covers the background and theoretical foundation of this work. Additionally, a theoretical example case in \autoref{subsec:background:theroy_example} is used to illustrate the expected performance behaviour of the WebArgo platform. \autoref{sec:background:related_work} presents related works to describes the state of research and development in the field of volunteer computing and edge computing with WebAssembly.

\section{Distributed Computing \& Volunteer Computing}
Distributed computing describes a architecture where many autonomous computing elements collaborate as a unified system to solve complex computational problems through coordinated sharing of resources. A distributed computing system typically consists of multiple independent nodes, each possessing local memory and computational resources, interconnected through a network infrastructure that facilitates communication and synchronization between processes. This concept is particularly suitable to compute parallelizable workloads.

Volunteer Computing is a sub category of distributed computing that focuses on public support. Participants share their computer processing power to support a collective interest. The idea of volunteer computing already emerged in 1996 \cite{relatedwork:boinc1} and \citeauthor{background:vcname} later characterized the term \emph{volunteer computing} \cite{background:vcname}. The first large-scale scientific projects SETI@home and Folding@Home where launched in 1999 \cite{relatedwork:boinc1,relatedwork:seti}. These pioneering projects are since than demonstrating the success and popularity of volunteer computing.

Since the total amount of edge devices like laptops, smartphones, tablets and desktop computers continues to increase \cite{background:amountdeviceses,relatedwork:boinc1} there is a big pool of devices that can be used for volunteer computing. A study in 2014 has estimated that about 2 Billion computers are actively in use worldwide \cite{intro:computersAmount}. However, many of these resources often remain idle or underutilized during regular operation \cite{relatedwork:mobilecloud, relatedwork:wasmedgecomputing}. While these untapped resources may seem insignificant individually, they collectively represent substantial potential due to their total number. These available resources located in office buildings, universities or public spaces can be utilized to participate in volunteer computing projects and thereby share their unused computational capabilities.

Furthermore, a study of 2011 revealed significant potential acceptance of volunteer computing in society, with 37\% of respondents indicating they were either \emph{"somewhat likely"} or \emph{"very likely"} to participate in volunteer computing projects \cite{intro:volunteerStudy}. However, the study also identified privacy and security concerns as primary factors that negatively influence willingness to volunteer in such projects \cite{intro:volunteerStudy}.

\section{WebAssembly}
\label{sec:background:webassembly}
(TODO) Introduce WebAssembly as key and enabeling technologies - enhance Section ?

WebAssembly \cite{methodology:wasmW3C} is a key technology utilized in the implementation of WebArgo, offering several advantages for distributed or edge computing \cite{relatedwork:wasmedgecomputing}. It provides near-native performance, while maintaining platform independence \cite{methodology:wasm, methodology:wasmW3C, relatedwork:wasmedgecomputing}. 

WebAssembly is a low-level binary instruction format designed to serve as a compilation target for high-level programming languages \cite{methodology:wasm, methodology:wasmW3C, methodology:wasm2}, that promises near-native performance execution in web browsers \cite{methodology:wasm, methodology:wasmW3C, relatedwork:wasmedgecomputing}. It employs a stack-based virtual machine architecture, operates in a memory-safe sandboxed environment, and interacts with JavaScript of the browser environment through a defined \ac{API} \cite{methodology:wasm, methodology:wasmW3C, methodology:wasm2, methodology:wasmdocu}. A compiled WebAssembly is particularly effective for computationally intensive tasks in the browser \cite{methodology:wasm2, methodology:wasmW3C} like image processing, game engines, or cryptographic operations.

The flexible features of platform independence and support of multiple high-level programming languages makes WebAssembly a fantastic choice for the WebArgo platfrom.
\\~\\
Furthermore, WebAssembly is maintaining the browsers inherent security model \cite{methodology:wasmW3C, methodology:wasm2, methodology:wasmdocu}. This is preventing any application to read and write files or to access device hardware such as cameras or microphones without explicit user permissions. Therefore WebArgo is ensuring a higher security for worker nodes than other volunteer computing applications, which require the installation of third-party software directly to the machine. This directly addresses the privacy and security concerns revealed by the study of \citeauthor{intro:volunteerStudy} \cite{intro:volunteerStudy} to increase the potential pool of participating workers.

\section{Related Work}
(TODO) bla bla

\subsection{Volunteer Computing}
The idea of volunteer computing has been established for a long time. The earliest concept of volunteer computing started already back in 1996 \cite{relatedwork:boinc1}. This section introduces relevant projects in this field, which have been used in the past or are currently in active use and maintained.

\subsubsection{SETI@home}
\label{subsec:background:related_work:seti}
\ac{SETI} at home (\ac{SETI}@home) represents a pioneering project in the field of distributed and volunteer computing. It utilises the resources of volunteer computers worldwide to analyze radio telescope data in search of extraterrestrial intelligence. Data from the Arecibo radio telescope is divided into smaller chunks which are distributed to volunteer computers for processing. The system is designed to identify several types of patterns in radio signals that could indicate an artificial origin. These anormalies are spikes, Gaussians, pulsed signals, and triplets in the collected radio data. \cite{relatedwork:seti}

In the year 2002, \ac{SETI}@home had engaged over 3.8 million participants across 226 countries, achieving an average computing power of 27.36 teraFLOPS \cite{relatedwork:seti}.

\ac{SETI}@home demonstrated the viability of large-scale public-resource computing and identified key factors that make tasks suitable for this approach:
\begin{itemize}
  \item High computing-to-data ratio
  \item Independent parallelism
  \item Error tolerance
  \item Ability to attract public interest
\end{itemize}
The success of \ac{SETI}@home not only advanced scientific research but also increased public awareness and involvement in scientific participation. This project laid therefore the groundwork for future volunteer computing projects and frameworks. \cite{relatedwork:seti}

\subsubsection{XtremWeb}
\label{subsec:background:related_work:xtremweb}
XtremWeb is an experimental Global Computing platform designed to harness idle computing resources connected to the internet for high-throughput computing. The system aims to provide a platform for experimenting with Global Computing capabilities and addressing issues such as scalability, heterogeneity, availability, fault tolerance, security, and usability in massively distributed computing environments. \cite{relatedwork:xtremweb}

The architecture of XtremWeb consists of three main components:
\begin{enumerate}
  \item Workers: These are volunteer PCs that execute tasks. They are implemented primarily in Java for portability, with native code used for OS-specific functions. Workers monitor resource availability based on user-defined policies and support multiple platforms, including Linux and Windows.
  \item Servers: These manage tasks and applications. The server design is modular, with components for application pool, job pool, accounting, and scheduling. Servers can be clustered for increased throughput and support specialization for tasks such as dedicated result collection.
  \item Clients: These submit tasks to the system.
\end{enumerate}
XtremWeb employs a two-part scheduling system:
\begin{enumerate}
  \item Dispatcher: Selects tasks from the pool based on application priorities.
  \item Scheduler: Assigns tasks to workers.
\end{enumerate}
The default scheduling policy of XtremWeb is \ac{FIFO} and XtremWeb implements a timeout to enable rescheduling of aborted tasks \cite{relatedwork:xtremweb}.

Furthermore XtremWeb implements a worker-initiated communication protocol, which facilitates easier deployment through firewalls. This protocol consists of the four main requests: \emph{hostRegister}, \emph{workRequest}, \emph{workAlive}, and \emph{workResult}. \cite{relatedwork:xtremweb}

XtremWeb has been successfully applied to various projects in the past. Thes include the Pierre Auger Observatory for studying high-energy cosmic rays. In this application, XtremWeb was used to run the \ac{AIRES} program by partitioning large simulations into smaller subtasks. \cite{relatedwork:xtremweb}

In summary, XtremWeb provides a flexible and robust platform for experimenting with Global Computing concepts. Its approach and concept has some similarities to WebArgo. The original XtremWeb project seems not to be maintained since its development in the early 2000s.

\subsubsection{EGEE}
\label{subsec:background:related_work:egee}
Initiated on April 1, 2004, \ac{EGEE} was a two-year project within a planned four-year program, involving 71 partners from Europe, Russia, and the United States \cite{relatedwork:egee}. The project's primary objectives were to establish a seamless European grid infrastructure for scientific research, provide production-level grid services, and re-engineer grid middleware for enhanced robustness and scalability \cite{relatedwork:egee}.

The infrastructure of \ac{EGEE} was built upon the EU Research Network GEANT, leveraging expertise from previous initiatives such as EU DataGrid, UK e-Science, INFN Grid, Nordugrid, and US Trillium. The project aimed to expand its computational capabilities from an initial 3000 \acs{CPU}s at 10 sites after the first month to 10000 \acs{CPU}s across 50 sites by the end of the second active year \cite{relatedwork:egee}.

\ac{EGEE} focused on two primary pilot applications: 
\begin{itemize}
  \item The \ac{LCG} for high-energy physics data analysis.
  \item Biomedical Grids addressing challenges in genomic database mining and medical database indexing
\end{itemize}
The project focused on re-engineering existing middleware to address issues from first-generation implementations and ensure adherence to \ac{OGSA} standards. This approach aimed to enhance the reliability and scalability of the grid infrastructure \cite{relatedwork:egee}.

In conclusion, the \ac{EGEE} project represented a significant effort to transform grid computing from a research concept into a practical infrastructure to support "e-science" across Europe.

\subsubsection{BOINC}
\label{subsec:background:related_work:boinc}
(TODO) More about BOINC and compare BOINC to WebArgo | Gegenüberstellung

\ac{BOINC} is an open-source middleware system for volunteer computing, which utilizes consumer devices for scientific computing. It addresses key challenges in distributed computing, such as dealing with untrusted, unreliable, and heterogeneous computing resources, validating results from potentially malicious hosts, and supporting diverse applications and computing environments. \cite{relatedwork:boinc1}

The BOINC architecture consists of server components, including a database and job scheduler, and client software comprising a core client, GUI, and screensaver. Communication between clients and servers is facilitated through HTTP-based RPCs. BOINC employs replication-based validation and adaptive replication to ensure result integrity while minimizing overhead. The system also implements sophisticated scheduling policies both on the client and server sides to optimize resource allocation and job dispatch. \cite{relatedwork:boinc1}

BOINC has been successful in supporting numerous scientific projects and providing substantial computing power. \cite{relatedwork:boinc1}

\subsection{WebAssembly for Edge Computing}
\label{subsec:background:related_work:wasmedgecomputing}
(TODO) Gegenüberstellung zu WebArgo - WebArgo nutzt quasi WebAssembly as Edgecomputing

The work from \citeauthor{relatedwork:wasmedgecomputing} \cite{relatedwork:wasmedgecomputing} investigates WebAssembly as a promising solution for edge computing challenges, particularly addressing the crucial requirements of portability and migratability for such applications. This work builds upon previous research in edge computing virtualization while introducing novel perspectives on using WebAssembly as an enabling technology for portable and migratable edge applications. Their comprehensive analysis demonstrates WebAssembly's capabilities in achieving near-native performance while maintaining platform independence, which proposes a significant advancement compared to traditional virtualization approaches. In their work the authors present a systematic evaluation of different WebAssembly execution environments, including browser-based implementations and standalone runtimes, focusing on the performance to portability trade-off. 

Their findings highlight WebAssembly's potential as a lightweight alternative to conventional virtual machines or container based approaches in edge computing scenarios \cite{relatedwork:wasmedgecomputing}. However, they identified that WebAssembly is currently lacking migratability features, compared to virtual machines and containers, to make it a complete solution for edge computing environments \cite{relatedwork:wasmedgecomputing}. Depending on the migration scenario they provide ideas for further actions or development to utilize WebAssembly.

\subsection{Dynamic Mobile Device Clusters in
Edge Femtoclouds}
(TODO) Gegenüberstellung zu WebArgo - Heterogenous Clients enabled, Easy Job Distribution, not a prototype (=Security, Persistent, Scheduling), easy deployment \& participation

The work from \citeauthor{relatedwork:mobilecloud} \cite{relatedwork:mobilecloud} explored the concept of Edge Femtoclouds, which demonstrated significant performance improvements through collaborative mobile edge computing \cite{relatedwork:mobilecloud}. Their experimental results show that Edge Femtoclouds can reduce job completion time by up to 26\% compared to traditional cloud computing approaches, with their novel checkpointing mechanism further improving efficiency by an additional 31\% \cite{relatedwork:mobilecloud}. The authors developed a comprehensive system architecture incorporating cloud-based controllers, mobile device helpers, and job managers, successfully addressing the challenges of device churn and system stability. 

Their prototype demonstrated robust performance across varying job arrival rates \cite{relatedwork:mobilecloud}. While highlighting the benefits of reduced latency and network congestion, their findings also identified key challenges including the need for sufficient device participation and robust security measures \cite{relatedwork:mobilecloud}. The work presents Edge Femtoclouds as a cost-effective alternative to dedicated edge servers, though noting the importance of developing appropriate incentive mechanisms for device owners.

They stated that the support of heterogen devices is a important feature for such a application, due to the heterogeneous characteristic of the mobile device environment \cite{relatedwork:mobilecloud}. However the implemented prototype that was built in their work was only supporting android devices as a proof of concept.