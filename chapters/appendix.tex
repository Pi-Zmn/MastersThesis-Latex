\chapter{Appendix}

\section{System Specifications}

\subsection{Linux System}
\label{app:system:mymachine}
Tests and code executions have been made on a system with the following specifications:

\begin{itemize}
    \item \textbf{CPU:} Intel(R) Core(TM) i5-8250U CPU @ 1.60GHz
    \item \textbf{RAM:} 15 Gi (15.4667 GB)
    \item \textbf{Operating System:} Ubuntu 20.04.6 LTS
    \item \textbf{Kernel:} 5.15.0-119-generic
    \item \textbf{GPU:} Intel Corporation UHD Graphics 620 (rev 07)
\end{itemize}

\section{Sourcecode and Output}

\subsection{Calculation of Mandelbrot Set: C++}
\label{app:code:mandelbrot1}
Below is the C++ source code for calculating the Mandelbrot set:

\begin{lstlisting}[language=C++, frame=tb, caption={Mandelbrot Set Calculation}]
    #include <iostream>
    #include <complex>
    #include <chrono>
    
    using namespace std;
    using namespace std::chrono;
    
    // Function to compute Mandelbrot iterations for a given complex number
    int mandelbrot(complex<long double> c, int max_iter) {
        complex<long double> z = 0;
        int n = 0;
    
        while (abs(z) <= 2.0 && n < max_iter) {
            z = z * z + c;
            n++;
        }
    
        return n;
    }
    
    // Mandelbrot calculation over a 2D area defined by xmin, xmax, ymin, ymax
    void calculate_mandelbrot(long double xmin, long double xmax, long double ymin, long double ymax, int width, int height, int max_iter) {
        for (int x = 0; x < width; ++x) {
            for (int y = 0; y < height; ++y) {
                // Map pixel position to complex plane
                long double real = xmin + (x / (long double)width) * (xmax - xmin);
                long double imag = ymin + (y / (long double)height) * (ymax - ymin);
                complex<long double> c(real, imag);
    
                mandelbrot(c, max_iter);
            }
        }
    }
    
    int main(int argc, char* argv[]) {
        if (argc != 5) {
            cerr << "Usage: " << argv[0] << " <xmin> <xmax> <ymin> <ymax>" << endl;
            return 1;
        }
    
        // Parse command-line arguments for region bounds
        long double xmin = atof(argv[1]);
        long double xmax = atof(argv[2]);
        long double ymin = atof(argv[3]);
        long double ymax = atof(argv[4]);
    
        // Settings for calculation
        const int width = 512;     // Resolution width
        const int height = 512;    // Resolution height
        const int max_iter = 100;  // Maximum number of iterations
    
        // Start measuring time
        auto start = high_resolution_clock::now();
    
        // Calculate Mandelbrot set in the specified region
        calculate_mandelbrot(xmin, xmax, ymin, ymax, width, height, max_iter);
    
        // Stop measuring time
        auto end = high_resolution_clock::now();
    
        // Print Computing Time
        cout << "Mandelbrot calculation completed in " << duration_cast<milliseconds>(end - start).count() << "[ms]" << endl;
        cout << "Mandelbrot calculation completed in " << duration_cast<seconds>(end - start).count() << "[s]" << endl;
    
        return 0;
    }
\end{lstlisting}

This code has been executed in three isolated runs with following input parameters:
\begin{lstlisting}
    ./mandelbrot -2.0 1.0 -1.5 1.5
\end{lstlisting}

The output of this code on the system specified in \ref{app:system:mymachine} was:
\begin{itemize}
    \item \begin{lstlisting}
        Mandelbrot calculation completed in 620[ms]
        Mandelbrot calculation completed in 0[s]
    \end{lstlisting}
    \item \begin{lstlisting}
        Mandelbrot calculation completed in 608[ms]
        Mandelbrot calculation completed in 0[s]
    \end{lstlisting}
    \item \begin{lstlisting}
        Mandelbrot calculation completed in 613[ms]
        Mandelbrot calculation completed in 0[s]
    \end{lstlisting}
\end{itemize}

\subsection{Calculation of Mandelbrot Set: C++ (WASM build)}
\label{app:code:mandelbrot2}
Below is the C++ source code for calculating the Mandelbrot set with all additions to be complied to .wasm with emscripten (emcc):

\begin{lstlisting}[language=C++, frame=tb, caption={Mandelbrot Set Calculation (WASM build)}]
#include <emscripten.h>
#include <iostream>
#include <complex>
#include <chrono>

using namespace std;
using namespace std::chrono;

// Function to compute Mandelbrot iterations for a given complex number
int mandelbrot(complex<long double> c, int max_iter) {
    complex<long double> z = 0;
    int n = 0;

    while (abs(z) <= 2.0 && n < max_iter) {
        z = z * z + c;
        n++;
    }

    return n;
}

// Mandelbrot calculation over a 2D area defined by xmin, xmax, ymin, ymax
void calculate_mandelbrot(long double xmin, long double xmax, long double ymin, long double ymax, int width, int height, int max_iter) {
    for (int x = 0; x < width; ++x) {
        for (int y = 0; y < height; ++y) {
            // Map pixel position to complex plane
            long double real = xmin + (x / (long double)width) * (xmax - xmin);
            long double imag = ymin + (y / (long double)height) * (ymax - ymin);
            complex<long double> c(real, imag);

            mandelbrot(c, max_iter);
        }
    }
}

EMSCRIPTEN_KEEPALIVE
void wasmMain(long double xmin, long double xmax, long double ymin, long double ymax) {
    // Settings for calculation
    const int width = 512;     // Resolution width
    const int height = 512;    // Resolution height
    const int max_iter = 100;  // Maximum number of iterations

    // Start measuring time
    auto start = high_resolution_clock::now();

    // Calculate Mandelbrot set in the specified region
    calculate_mandelbrot(xmin, xmax, ymin, ymax, width, height, max_iter);

    // Stop measuring time
    auto end = high_resolution_clock::now();

    // Print Computing Time
    cout << "Mandelbrot calculation completed in " << duration_cast<milliseconds>(end - start).count() << "[ms]" << endl;
    cout << "Mandelbrot calculation completed in " << duration_cast<seconds>(end - start).count() << "[s]" << endl;
}

int main(int argc, char *argv[]) {
    // Set default region bounds
    long double xmin = -2.0;
    long double xmax = 1.0;
    long double ymin = -1.5;
    long double ymax = 1.5;

    // Parse command-line arguments for region bounds
    if (argc == 5) {
        xmin = atof(argv[1]);
	xmax = atof(argv[2]);
	ymin = atof(argv[3]);
	ymax = atof(argv[4]);
    }

    wasmMain(xmin, xmax, ymin, ymax);

    return 0;
}
\end{lstlisting}

\subsection{Generating Plots: Python}
\label{app:code:theoryplots}

The following code was used to generate figure \ref{fig:background:theoryA}:
\begin{lstlisting}[language=Python, frame=tb, caption={Ploting execution time (variable nodes)}]
    import seaborn as sns
    import matplotlib.pyplot as plt
    import math
    import numpy as np
    
    t = 20
    n = 5
    tc = 614
    tl = 32
    setup_offset = 75
    
    # Run Time on single device
    threshold_value = (t * tc) / 1000
    
    # Time Data
    x = np.linspace(1, n, n)
    # Run Time distributed on n devices
    y = np.array([((math.ceil(t / n_val) * (tc + tl)) + setup_offset) / 1000 for n_val in x])
    
    # Create a line plot for the array of values
    sns.set(style="whitegrid")
    plt.plot(x, y, label="Runtime on N devices", color="blue")
    
    # Add a horizontal dotted line for the threshold value
    plt.axhline(y=threshold_value, color='red', linestyle='--',
                label=f'Runtime on single device ({threshold_value}s)')
    
    # Set plot title and labels
    plt.title("Impact of varying nodes N on total runtime\n"
              f'({t} Tasks, {tc}ms Computation time, {tl}ms Round Trip Time, {setup_offset}ms Offset)')
    plt.xlabel("Number of nodes N")
    plt.ylabel("Total runtime in Seconds [s]")
    
    # Show legend
    plt.legend()
    
    # Save the plot as an SVG file
    plt.savefig("line_plot_with_threshold.svg", format="svg")
    
    # Show the plot
    plt.show()
\end{lstlisting}

The following code was used to generate figure \ref{fig:background:theoryB}:
\begin{lstlisting}[language=Python, frame=tb, caption={Ploting execution time (variable tasks)}]
    import seaborn as sns
    import matplotlib.pyplot as plt
    import math
    import numpy as np
    
    t = 15
    n = 3
    tc = 614
    tl = 32
    setup_offset = 75
    
    # Task Container
    x = np.linspace(1, t, t)
    
    # Run Time on single device
    threshold_value = np.array([(t_val * tc) / 1000 for t_val in x])
    
    
    # Run Time distributed on n devices
    y = np.array([((math.ceil(t_val / n) * (tc + tl)) + setup_offset) / 1000 for t_val in x])
    
    # Create a line plot for the array of values
    sns.set(style="whitegrid")
    plt.plot(x, y, label="Runtime on N devices", color="blue")
    
    # Add a dotted line for the threshold value
    plt.plot(x, threshold_value, color='red', linestyle='--', label=f'Runtime on single device')
    
    # Set plot title and labels
    plt.title("Impact of varying ammount of tasks T on total runtime\n"
              f'({n} Nodes, {tc}ms Computation time, {tl}ms Round Trip Time, {setup_offset}ms Offset)')
    plt.xlabel("Number of tasks T")
    plt.ylabel("Total runtime in Seconds [s]")
    
    # Show legend
    plt.legend()
    
    # Save the plot as an SVG file
    plt.savefig("line_plot_with_threshold.svg", format="svg")
    
    # Show the plot
    plt.show()
\end{lstlisting}