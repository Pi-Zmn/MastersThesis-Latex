\chapter{Introduction}
\label{ch:intro}
The work of this paper proposes a new approach to implement a volunteer computing platform. WebCrowd leverages the infrastructure of the continuously evolving web ecosystem. While the modern web provides multiple technologies to develop a robust and high-performant volunteer computing platform, the continuous development of hardware for consumer devices has led to a significant increase of computational capabilities at the edge of the internet \cite{relatedwork:mobilecloud, relatedwork:wasmedgecomputing}. The key web technologies utilized in the implementation of WebCrowd include:
\begin{itemize}
    \item \textbf{WebAssembly} \cite{methodology:wasmW3C}, a binary format that can be executed in a web browser. Multiple programming languages support WebAssembly as a compilation target and it is promissing near-native performance while providing platform independent code execution \cite{methodology:wasm, methodology:wasmW3C}.
    \item \textbf{WebSockets} \cite{methodology:websockets1}, a bidirectional communication chanel that establishes persistent connections between clients and servers. This technology enables data transmission in both directions while maintaining a minimal overhead. Therfore WebSockets significantly reduce the latency compared to traditional \acs{HTTP} polling methods \cite{methodology:websockets3}. 
    \item \textbf{WebWorker} \cite{methodology:webworkers}, enabling parallel processing in the browser by executing JavaScript code in isolated background threads. These WebWorker threads are independent of the main browser UI thread and prevent computationally intensive tasks from blocking the user interface rendering process \cite{methodology:webworkers}.
\end{itemize}
Edge devices that participate in volunteer computing are usually very diverse in hardware and operating system \cite{intro:diverseDevices}. Furthermore, these clients are most likely unreliable \cite{relatedwork:boinc1}, as they fluctuate in availability and may disconnect during active task execution. To manage such an inconsistent crowd of heterogeneous clients, WebCrowd implements the following two main features:
\begin{enumerate}
    \item \textbf{Heterogeneity:} The platform supports a diverse range of devices, regardless of their operating system or hardware. The only requirement to participate in a job distributed by WebCrowd is access to a modern browser, which supports the previously specified web technologies. Supported client devices range from personal computers and smart-devices to single-board computers.
    \item \textbf{Dynamic participation:} The amount of participating devices is variable and can change dynamically during the runtime of a job. The implemented infrastructure of WebCrowd ensures that all active jobs are completed without issues, even as the number of participating devices fluctuates. Hence clients are able to join an active job or disconnect from the platform at any time without disrupting ongoing processes.
\end{enumerate}

\section{Motivation}
\label{sec:intro:motivation}
The primary objective of this work is to develop a resource-efficient, accessible and flexible alternative to conventional data centers and \ac{HPC} infrastructures. These traditional \ac{HPC} infrastructures often require significant investments in dedicated hardware and maintenance. This creates barriers for universities, research projects or smaller businesses that do not have access to external supercomputers and do not possess the necessary resources to build a \ac{HPC} infrastructure themselves. WebCrowd aims to provide for such organizations with limited resources a simple and user friendly platform to execute extensive research or development tasks. By leveraging WebCrowd, these organizations can establish their own local volunteer computing network that utilizes existing and available computational resources. Furthermore, WebCrowd can be expanded to a global volunteer computing platform where volunteering clients are able to support research projects and jobs world wide.

\section{Objectives}
\label{sec:intro:objectives}
WebCrowd aims to reduce the overhead for volunteering participants, thereby beeing more appealing for clients and potentially generating a larger volunteer base. The platform is designed in a way that volunteers require no setup to participate in projects. They connect to the web application through their browser via \acs{URL} and automatically participate in the active job of the volunteer computing network, with the platform's structure autonomously handling all processes in the background. No files or executables need to be downloaded on to the device, except for a modern browser, which already comes pre-installed on most consumer devices. Volunteers can transparently monitor the computing process of tasks through the web interface and maintain flexibility in joining or leaving the process at any time. When participants disconnect from the platform, no persistent files or data remain stored on their devices leaving no permanent effect on the device.

A study has revealed that privacy and security concerns significantly impact willingness to participate in volunteer computing projects \cite{intro:volunteerStudy}. The approach to utilize the browser environment offers an additional advantage in addressing these concerns, as it implements an inherent security model. The browser cannot read or write files or access device hardware such as cameras or microphones without explicit user permissions. WebAssembly, the key technology underlying WebCrowd, is also restricted to these security constraints. Additionally, the fact that volunteers are not required to install any third-party tool on their device could enhance the trust and positively effect the sense of security for participants.
\\~\\
Furthermore, WebCrowd aims to minimize any overhead for administrators that maintain the platform or develop additional jobs. The process of creating and distributing new jobs through the platform is designed to be straightforward and does not require a deep understanding of the underlying architecture of WebCrowd. WebAssembly significantly facilitates this objective, since it is designed to be a compilation target for numerous high-level programming languages \cite{methodology:wasm, methodology:wasmW3C, methodology:wasmdocu, relatedwork:wasmedgecomputing}. Developers can select a programming language with which they are familiar to create projects, while the overhead required to implement this code into the WebCrowd framework remains comparatively low. 

In addition, administrators are able to monitor and controll the deployed volunteer computing platform through a transparent web application that implements specific security measures to be accessible only by an administrator.

\subsection{Research Questions}
\label{subsec:into:objectives:questions}
This work investigates the viability of the developed volunteer computing platform - WebCrowd. Through empirical testing, this evaluation focuses on addressing three fundamental research questions that examine WebCrowd's computational capabilities, dynamic adaptability, and support of heterogeneous devices:
\begin{enumerate}
    \item \textbf{Computational Capability:} Is WebCrowd capable of successfully solving large, parallelizable problems?
    \item \textbf{Dynamic Viability:} Is the WebCrowd platform stable in a environment with dynamic clients?
    \item \textbf{Heterogeneous Viability:} Does WebCrowd support a diverse range of client devices without issues?
\end{enumerate}
These research questions are designed to evaluate critical aspects of WebCrowd's functionality and performance. The first question examines the platform's fundamental ability to successfully distribute task across the network to complete a parallelizable job. The second question addresses the platform's resilience and stability when dealing with varying numbers of connected clients and potentially unreliable connections. The third question investigates the platform's compatibility across different device hardware and operating systems. Through these investigations, this work aims to validate WebCrowd's effectiveness as a viable solution for distributed computing in the modern web environment.

\section{Structure}
\label{sec:intro:structure}
This paper presents a comprehensive examination of the developed WebCrowd platform. The foundation is established in \autoref{ch:background}, which explores the fundamental concepts of volunteer computing, provides the theoretical background, and examines related work in the fields of distributed and volunteer computing as well as edge computing with WebAssembly. \autoref{ch:methodology} introduces the key technologys as well as the structure of the development process. The implementation process and challenges that occured during the development are discussed in \autoref{ch:implementation}, providing insights into the architecture and technical desing of the platform. \autoref{ch:evaluation} describes the experimental setup that has been used to benchmark the platform and evaluates the corresponding results of these benchmarks. The paper concludes in \autoref{ch:conclusion} with a summary of the findings and outlines potential future research directions of this project.