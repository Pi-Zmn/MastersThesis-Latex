\chapter{Methodology}
\label{ch:methodology}
This chapter enumerates and describes all the technologies, frameworks, and tools utilized in the development of the volunteer computing platform, along with the reasoning for their selection. Each section provides a detailed explanation of these components, establishing a foundation for the following chapter, \ref{ch:implementation} Implementation. Additionally, all entities are listed and described in section \ref{sec:methodology:entities} to define the specific terms used in the subsequent chapters.

\section{Entities}
\label{sec:methodology:entities}
This section describes all entities and establishes the terms used to describe the processes in this work.

\subsection{Job}
\label{subsec:methodology:entities:job}
The Job entity represents a problem to be processed or solved using the platform. The platform is designed to handle multiple Jobs and each Job is a unique object. The State of a Job can have one of the following values:

\begin{itemize}
\item PENDING
\item ACTIVE
\item RUNNING
\item STOPPED
\item DONE
\end{itemize}

Figure \ref{fig:methodology:job-task} illustrates the \ac{UML} class diagram for the Job entity on the left side, lisitng all important attributes and methods of a Job. The language attributes defines the programming langugae of the source code that has been compiled to a WebAssembly binary file.

\begin{figure}[htbp]
  \centering
  \includegraphics[width=0.75\textwidth]{gfx/figures/Job-Task.png}
  \caption{\ac{UML} Class Diagram: Job \& Task}
  \label{fig:methodology:job-task}
\end{figure}

\subsection{Task}
\label{subsec:methodology:entities:task}
Each Job is divided into multiple equally sized Tasks. These Tasks are distributed to the worker nodes of the platform. Tasks are unique objects, each holding the specific input parameters that describe the particular portion of the Job to which the Task is assigned. Figure \ref{fig:methodology:job-task} displays the \ac{UML} class diagram for the Task entity on the right side. The relationship between Job and Task entities is defined as One-to-Many, meaning a Job can consist of multiple Tasks, but each Task is always assigned to a single Job.

\subsubsection{Batch}
\label{ssubsec:methodology:entities:task:batch}
A Batch is defined to be a subset of Tasks, wich are all assigned to the same Job.

\subsection{Client}
\label{subsec:methodology:entities:client}
A Client represents a node that accesses the platform. The actions a Client can perform are determined by the User Role assigned to its User. This User Role can have one of the following values:
\begin{itemize}
  \item User
  \item Admin
\end{itemize}
Figure \ref{fig:methodology:client} displays the \ac{UML} class diagram for the User entity at the top of the illustration.

\begin{figure}[htbp]
  \centering
  \includegraphics[width=0.75\textwidth]{gfx/figures/Client.png}
  \caption{\ac{UML} Class Diagram: User - Worker \& Administrator}
  \label{fig:methodology:client}
\end{figure}

\subsubsection{Worker}
\label{ssubsec:methodology:entities:client:worker}
Workers are Clients that voluntarily donate their computing power to support an active Job. Therfore they receive Tasks from the current Batch sent by the Server (\ref{subsec:methodology:entities:Server}), compute these Tasks, and return the corresponding results to the Server. Figure \ref{fig:methodology:client} presents the \ac{UML} class diagram of the Worker entity on the left side, listing all relevant attributes and methods. As Workers connect to the Server via web browsers, the browser's User Agent is utilized to characterize the Worker object. A browser User Agent can provide usefull information about the device's hardware resources and operating system.

\subsubsection{Administrator}
\label{ssubsec:methodology:entities:client:admin}
An Administrator Client can perform the same actions as a Worker, but also has additional capabilities. Administrators have the ability to set the state of Jobs and to oversee the progress of all Jobs. Figure \ref{fig:methodology:client} illustrates the \ac{UML} class diagram of the Administrator entity on the right side.

\subsection{Server}
\label{subsec:methodology:entities:Server}
The Server is used as the central component to maintain the network. Each Client intending to connect to the network must establish a connection with the Server.

Additionally the Server is responsible for scheduling and distributing the Tasks to all Workers. It also persistently stores each Job along with all results of its corresponding Tasks.

\section{WebAssembly}
\label{sec:methodology:wasm}
WebAssembly is a key technology in this proposed system, offering several advantages for distributed computing. It provides near-native performance, platform independence, and potential multi-threading capabilities. WebAssembly allows to compile code written in languages such as C, C++, and Rust to a binary format that can be executed in web browsers, enabling efficient computation on diverse client devices.

\cite{methodology:wasmdocu, relatedwork:wasmedgecomputing}
\subsection{emscriptren for C and C++}
\label{subsec:methodology:wasm:cpp}
compiler from C \& C++ to .wasm with .js Gluecode (setup wasm environment) \cite{methodology:emcc}
\subsection{Go}
\label{subsec:methodology:wasm:go}
compiler from Go to .wasm, offically from Go \cite{methodology:go}
\subsection{Pyodie for Python}
\label{subsec:methodology:wasm:python}
"Pyodide is a port of CPython to WebAssembly/Emscripten". \cite{methodology:pyodie}

\section{WebSokets}
\label{sec:methodology:websokets}
WebSockets play a crucial role in this system's communication infrastructure. Unlike traditional HTTP requests, WebSockets enable faster, bidirectional communication between clients and servers. This real-time capability is essential for efficient task distribution, progress monitoring, and result collection in the distributed computing platform.

\cite{methodology:websockets1, methodology:websockets2, methodology:websockets3}

\section{WebWorker}
\label{sec:methodology:webworker}
Web Workers are employed to enhance the performance and responsiveness of client-side computations. By executing computationally intensive tasks in separate threads, Web Workers prevent blocking the main thread responsible for the user interface and other critical browser functions. This approach allows our system to utilize client resources more effectively while maintaining a smooth user experience.

\cite{methodology:webworkers}

\section{Frameworks}
\label{sec:methodology:frameworks}
This section introduces the frameworks selected for the development of the web application. The following criteria were used to guide the selection of a suitable framework for the backend as well as the frontend:
\begin{itemize}
    \item The framework is well-tested and provides a stable \ac{LTS} version.
    \item The framework is popular among web developers.
    \item I am familiar with the programming language.
\end{itemize}

\subsection{Backend}
\label{subsec:methodology:frameworks:backend}
It was crucial for the backend framework to be popular among web developers. Working with a popular framework improves the development process by ensuring the availability of extensive educational and support resources online. Additionally, a widely used framework increases the likelihood that the platform can be maintained or further extended by other programmers.
\begin{figure}[htbp]
 \centering
 \includegraphics[width=0.95\textwidth]{gfx/figures/Popular_BE.png}
 \caption{Most Popular Backend Frameworks (Jan 2024) by GitHub Stars \cite{backend:popularity}}
 \label{fig:methodology:popularBE}
\end{figure}
The 15 most popular backend frameworks of January 2024 are displayed in figure \ref{fig:methodology:popularBE}. The popularity for each framework of this list is calculated by the number of GitHub Stars from repositories listed in a GitHub Archive \cite{backend:popularity}. The selection options of the backend framework where based on this popularity list. 
\begin{figure}[htbp]
  \myfloatalign
  \subfloat[Requests/Second (2024-06-25) \cite{backend:benchmark1}]{
     \label{fig:methodology:benchmark1BE}
     \includegraphics[width=.95\linewidth]{gfx/figures/Benchmark1_BE.png}
   } \\
   \subfloat[Best fortunes responses per second (2023-10-17) \cite{backend:benchmark2}]{
     \label{fig:methodology:benchmark2BE}
     \includegraphics[width=.95\linewidth]{gfx/figures/Benchmark2_BE.png}
   }
   \caption{Most Popular Backend Frameworks from Figure \ref{fig:methodology:popularBE} ranked by Performance.}
   \label{fig:methodology:benchmarkBE}
\end{figure}
In addition to the previously stated criteria, the selected backend framework needed to meet specific performance requirements. It was essential for the framework to efficiently handle multiple connected clients with minimal latency. Furthermore, the framework's internal computation speed was critical, particularly for preparing input arguments for each task and managing task scheduling across all clients. The goal was to reduce overhead as much as possible to ensure high performance.

To identify a high-performance framework, the most popular backend frameworks listed in Figure \ref{fig:methodology:popularBE} were compared using two different benchmarks. The results of these benchmarks are presented in Figure \ref{fig:methodology:benchmarkBE}, where the frameworks are ranked from top to bottom based on their performance in both benchmarks. It is important to note that some frameworks from the list in Figure \ref{fig:methodology:popularBE} could not be found in these specific benchmarks. Both benchmark sources simulated numerous clients sending requests to each backend and measuring the number of successful responses per second in order to determine the performance \cite{backend:benchmark1, backend:benchmark2}.

Finally, NestJS \cite{methodology:nestjs} with Fastify was selected as the backend framework. It performed exceptionally well in both benchmarks shown in Figure \ref{fig:methodology:benchmarkBE} and is also a popular choice among developers.

\subsection{Frontend}
\label{subsec:methodology:frameworks:frontend}
The criteria state in \ref{sec:methodology:frameworks} also applied for the selection of a framework for the frontend. Figure \ref{fig:methodology:popularFE} shows the 15 most used frameworks among web developers in July 2024 \cite{frontend:popularity}. This statistic has been collected by statista and is a survey of Stack Overflow (TODO: Stack Overflow quelle und so) 

\begin{figure}[htbp]
  \centering
  \includegraphics[width=0.95\textwidth]{gfx/figures/Top15_Frameworks_2024_Statista.png}
  \caption{Most Used Web Frameworks in 2024 (Developer Survey on Stack Overflow) \cite{frontend:popularity}}
  \label{fig:methodology:popularFE}
\end{figure}

\section{Benchmark: Definition of Job}
\label{sec:methodology:benchmark}
How can i benchmark the platform? What problem / code is used to benchmark? Which hardware is used for the benchmark?