\chapter{Conclusion}
\label{ch:conclusion}
This work has introduced WebArgo, a volunteer computing platform that leverages several modern web technologies to create a new, dynamic and heterogeneous distributed computing solution. Through empirical evaluation, the implemented WebArgo platform has demonstrated its capability to effectively distribute computational workloads across heterogeneous client devices as workers while maintaining robustness in dynamic environments with fluctuating worker participation. 

All three research questions, listed in \autoref{subsec:into:objectives:questions}, were addressed in \autoref{ch:evaluation} and confirmed by the empirical evidence of the evaluations experiments. Furthermore, the experiments revealed that WebArgo is able to successfully reduce the total computation time of the Mandelbrot benchmark job, decribed in \autoref{sec:implementation:benchmark}, by 59\% compared to a native code execution, when distributing the corresponding tasks across four consumer devices found in a single household.

Additionally, WebArgo is successfully fulfilling all its objectives outlined in \autoref{sec:intro:objectives}. Hence, WebArgo provides an accessible volunteer computing solution that requires no setup except a browser and a internet connection for participants.  WebArgo also limits the overhead for administrators, because the development of new jobs does not require a deep understanding of the underlying architecture, and these jobs can be implemented in three different programming languages, through which C/C++ in combination with the emscripten \cite{methodology:emcc} toolchain has turned out to be the currently best performing option. Additionally, hosting a WebArgo instance has limited requirements and can potentially even be used by a single private individual. A public web server with enough storage for the incoming job results and capable of running Docker Compose \cite{conclusion:docker} to orchestrate the backend application, frontend application and database are enough to host a private WebArgo platform.

Furthermore, the web-based approach potentially addressed privacy and security concerns of participants, because installation and execution of third party applications is not required.
\\~\\
In conclusion, WebArgo successfully addresses key challenges in distributed computing while providing an accessible solution for organizations with limited access to traditional \ac{HPC} computing resources. Also its web-based approach reduces any barriers for participation while additionally maintaining the web security model and WebAssembly sandboxing, potentially enabling broader adoption of volunteer computing.


\section{Limitations}
\label{sec:conclusion:limitations}
While WebArgo demonstrates promising capabilities as a volunteer computing platform, several limitations have been identified during its development and evaluation that warrant acknowledgment. The current implementation of the scheduling algorithm operates on a simple \ac{FIFO} basis, without consideration of individual worker hardware capabilities. This approach potentially leads to suboptimal task distribution, as computationally intensive tasks may be assigned to less capable devices while more powerful workers remain underutilized. A more sophisticated scheduling mechanism that accounts for worker hardware specifications and historical performance metrics could improve overall system efficiency.

A significant performance limitation is rooted in the performance of WebAssembly. Despite WebAssembly is promised to perform at near native speed \citeauthor{background:not-so-fast} have discovered, that the WebAssembly execution time can potentially be twice as slow as the corresponding native code execution \cite{background:not-so-fast}, wich is consistent with measurements of this worker. This performance gap impacts WebArgo's computational efficiency. While the parallel processing capabilities of multiple distributed workers can compensate for this limitation, the inherent overhead remains a constraint on individual task performance.

Furthermore, the current implementation is not providing a real multi-threading option for task computation on a single worker device. This limitation prevents full utilization of available computational resources, as workers cannot leverage their potential multi-core capabilities. However, a worker with enough computational capabilitys can effectively execute multiple worker instances to process multiple tasks concurrently, each in an independent browser tap, as experiments of this work confirmed.

Currently, WebArgo's operational scope is constrained to support only one active job at a time. This restriction prevents workers to actively choose wich project or contributor they support with their participation, and therefore potentially reducing volunteer engagement.
\\~\\
These limitations are significant, but do not fundamentally undermine WebArgo's utility as a volunteer computing platform. In fact, these limitations represent opportunities for future development and enhancement of WebArgo.

\section{Future Work}
\label{sec:conclusion:future_work}
This section lists various drafts for future work options. The previously identified limitations of WebArgo provide a clear roadmap for future development and enhancement of the platform.  
\\~\\
\textbf{Performance-Aware Task Scheduling}
\newline
A potential focus for future development is the implementation of an intelligent scheduling algorithm that considers the hardware capabilities of participating workers and assumes the complexity of each task. Therefore, WebArgo could optimize the task distribution across the heterogeneous workers to ensure a optimal utilization of each individual worker.
\\~\\
\textbf{Multi-Threading Support}
\newline
Implementing a approach to enable multi-threaded task execution for workers represents a significant opportunity to further improve the performance WebArgo. This can be achieved either at the source code level by leveraging emscripten's Pthreads support \cite{methodology:emcc} for example, or at the browser level through the initialization of more than one WebWorker instances wich simultaneous process multiple tasks. This additon to WebArgo could be combined with the previously described performance-aware scheduling, hence scheduling multiple tasks for parallel processing to workers with sufficient hardware capabilities.
\\~\\
\textbf{Multi-Job Support}
Expanding WebArgo to support multiple actively running jobs simultaneously can improve the user experience for workers. To achieve this, the server needs to handle multiple batches of active tasks and the implemented communication protocol between server and worker needs to be updated. Furthermore, an additional feature to select preferred jobs can be included in the \ac{UI} of the client web page.  
\\~\\
\textbf{Extended Programming Language Support}
While WebArgo currently supports C++, Go, and Python, future work could focus on expanding WebAssembly compilation support for additional programming languages, since WebAssembly is a a compilation target for many more high-level programming languages \cite{methodology:wasm, methodology:wasmW3C, methodology:wasm2}. This can potentially make the platform more accessible to a broader range of developers and scientific computing applications.
\\~\\
\textbf{Implement the MapReduce model}
Implementing the MapReduce \cite{conclusion:map-reduce} model in WebArgo is a big opportunity to significantly enhance the platforms capability of providing more complex jobs. Currently the processing module of a job can be assumed to be similar to the mapping step. Each task therefore represents a single map operation. MapReduce could be used to aggregate all gathered task results of a job to a total result of the job. This enables the processing of more complex jobs like a single epoch of reinforcement learning, with each task representing an episode and the reduce step performing the corresponding policy updates. This approach can be further investigated to potentially enable reinforcement learning through a pipeline of multiple MapReduce jobs in WebArgo.