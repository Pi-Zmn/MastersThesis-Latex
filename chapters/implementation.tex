\chapter{Implementation}
\label{ch:implementation}
This chapter presents the technical implementation of the dynamic and heterogeneous volunteer computing platform WebArgo in detail. The implementation leverages modern web technologies introduced in \autoref{sec:methodology:technologies} to create a high-performance distributed computing solution for the web. At its core, the system utilizes WebAssembly for cross-platform compatibility, high-performance, and language independence (\autoref{sec:methodology:wasm}), WebSockets for efficient bidirectional communication (\autoref{sec:methodology:websockets}) and WebWorkers to enhance user experience and to enable potential parallel Task execution (\autoref{sec:methodology:webworker}).

The following sections describe WebArgo's communication protocol, persistence mechanisms, scheduling strategies, security measures and the challenges encountered during the implementation process. \autoref{sec:implementation:backend} and \autoref{sec:implementation:frontend} describe the implementation and usage of WebArgo's \ac{API} and web application interface.

\section{WebArgo's Communication Protocol}
\label{sec:implementation:communication}
When workers or administrators establish a connection to the platform by accessing the frontend application, a WebSocket connection to the backend is automatically initiated. This enables real-time and bidirectional communication between the backend and these clients. Therefore, this connection is used to send tasks from the backend to the workers as well to send the result of each task from the workers back to the backend. Furthermore, administrators receive real-time data about all connected workers and the current status of each job through their WebSocket connection.
\begin{figure}[htbp]
    \centering
    \includegraphics[width=0.9\textwidth]{gfx/figures/communication-connection-label.png}
    \caption{Communication: Connection of Worker \& Real-time Update for Administrator}
    \label{fig:implementation:communication1}
\end{figure}
~\\
The first sequence in \autoref{fig:implementation:communication1} illustrates the process of a worker establishing a connecting to the backend in WebArgo. Upon successful connection, the worker transmits all available information regarding its hardware and operating system in form of the browser user agent to the backend. After this initialization of the worker, all previously connected administrators automatically receive an updated list of all connected workers. Similarly, if a Worker disconnects a automatic update is send to all administrators, as described by the second sequence in \autoref{fig:implementation:communication1}.
\begin{figure}[htbp]
    \centering
    \includegraphics[width=0.95\textwidth]{gfx/figures/communication-jobexecution-label.png}
    \caption{Communication: Administrator Starts Job \& Worker Executes Tasks}
    \label{fig:implementation:communication2}
\end{figure}
~\\
\autoref{fig:implementation:communication2} illustrates the job initialization and execution process. The first sequence describes the job activation process and is initiated when an administrator changes a jobs status to \emph{ACTIVE}. The backend then generates a batch of tasks for this specific job and notifies all connected workers by transmitting a object of this activated job to them. Upon receiving this notification, each worker fetches the corresponding WebAssembly binary and JavaScript glue code file from the backend via \acs{HTTP} request to initializes a WebWorker with the specific WebAssembly environment for this currently active job. When this step is successfully completed the worker notifies the backend, which then marks this workers as \emph{ready}. This update is also forwarded to all administrators with an active WebSocket connection to the backend.

The second sequence in \autoref{fig:implementation:communication2} illustrates th process of starting an active job. This process is initiated when an administrator changes the status of an active job to \emph{RUNNING}. After that begins the backend to distribute unique tasks from the current batch to all workers who have been marked as \emph{ready} by transmitting the \emph{Worker Ready} message and therefore already have successfully completed the initialization process of the currently active job. Each worker executes its assigned task, then appends the corresponding result to the task object and transmits the completed task back to the backend application. If unprocessed or unscheduled tasks remain in the batch at the moment of receiving the completed task, the backend responds by assigning the next pending task to this worker. Additionally, the backend notifies all administrators after each successfully completed task about the current progress of the \emph{RUNNING} job. This enables monitoring of the jobs progression in real-time for all administrators.

If a Worker is connecting to WebArgo while there is already an active or running job, the communication sequence is automatically executed identically. Accordingly, the worker starts to initialize this job and then proceeds to process tasks of the corresponding batch, hence enabling dynamic participation in ongoing jobs for workers.
\\~\\
Each job has a timeout attribute for its tasks. Scheduled tasks that remain incomplete after their allocated timeout period can be redistributed to a different worker. This mechanism enables the rescheduling of tasks, which have been assigned to a malfunctioning worker or a worker that has disconnected before completing its assigned task. Additionally, this approach allows rescheduling of tasks that have been assigned to slower workers, known as stragglers, to more efficient workers. Hence, this mechanism can optimize the overall execution time of a job.
\begin{figure}[htbp]
    \centering
    \includegraphics[width=0.95\textwidth]{gfx/figures/communication-timeout.png}
    \caption{Communication: Rescheduling of Tasks after Timeout}
    \label{fig:implementation:communication3}
\end{figure}
~\\
\autoref{fig:implementation:communication3} illustrates the case, if all tasks of the batch already have been scheduled, but one or more are not yet completed. When a worker transmits a completed task to the backend - and therby is expecting to receive a new task to compute from the backend - while the timeout period of all other scheduled tasks has not expired, the backend responds with the \emph{No-Task} message accompanied by the job object of the currently active job. The worker extracts the status and the timeout value from this job objects and initiates a waiting period equivalent to the timeout value plus a randomized overhead, if the jobs status is still \emph{RUNNING}. This additional randomized overhead time prevents simultaneous task requests, if multiple workers are waiting for a task at the same time. After the completion of the waiting period, the worker transmits a \emph{Request Task} message to the backend. The backend responds either with a newly available task or another \emph{No-Task} message, repeating this sequence until the current batch is fully processed.

\section{Persistence}
\label{sec:implementation:persistence}
This section describes the persistent data storage implementation of the platform. The primary adressed objective is to ensure system recovery capabilities in cases of system failures, unexpected system crashes, or scheduled system restarts. The following data objects have been identified as critical for a system recovery:
\begin{itemize}
    \item Job data
    \item Task input arguments
    \item Task results
    \item User credentials
\end{itemize}
Information regarding workers, their hardware specifications and operating systems is intentionally excluded from permanent storage.
\\~\\
WebArgo implements a robust persistence mechanism for job progression. When a running job is stopped, the current progress and all task results are automatically saved, hence actively stopping a running job initiates the persistence process for this job's current state. This functionality should be utilized if the platform is scheduled to be restarted or terminated.

Additionally, the system persists the progress of a currently running job after each batch completion. This mechanism enables periodic job backups, as each batch completion establishes a save point. Consequently, the batch size determines the job's fallback tolerance in the event of an unexpected system failure.

It is noteworthy that from a worker's perspective, task progress can not be persisted. Since WebAssembly is executed in a save sandbox environment the code has usually no access to local files. In the event of an unexpected crash of the worker's WebAssembly process or browser, the progress of the affected task has not been persisted on the workers device and therefore will be lost.

\subsection{Data in Files}
The system utilizes text files for persistent storage of task input arguments and task results. Each job's critical information about its tasks is stored in a separate text file for task input arguments and the task results on the web server. 
~\\
The method to generate a new batch of the active job reads the corresponding task input arguments text file line by line, where each line represents an input argument. For each line, the method creates a unique task and adds it in sequence to the batch.

When the progress of a job is persisted, the backend writes the results of all completed tasks within the batch to the corresponding task results text file. Each result is written in a new line, and the results are stored in sequential order.
\\~\\
Additionally, the backend implements a special mechanism to handle tasks that produce files as results. When a worker transmits such a completed task, the backend immediately stores the result file in a result directory corresponding to the running job. Subsequently, when later a jobs progress is saved, the backend stores the file path to each result in the corresponding task results text file instead of the actual result data.

\subsection{Database}
The PostgreSQL \cite{methodology:db} database is used to store the user credentials as well as each job object. Each job object contains a progress attribute that represents a pointer indexing the last completed task in the task sequence. During job recovery, this progress value determines the starting point for the first task in the new batch.

Furthermore, the following subsections describe the implementation of all entities that are used in WebArgo, as introduced in \autoref{sec:concept:terminology}.

\subsubsection{Job}
The job entity represents a problem to be processed or solved using the WebArgo platform. The platform is designed to handle multiple jobs while each job is a unique object. The State of a job can have one of the following values:
\begin{itemize}
  \item PENDING
  \item ACTIVE
  \item RUNNING
  \item STOPPED
  \item DONE
\end{itemize}
However, at the current state of the WebArgo implementation only one job at a time can have the \emph{ACTIVE} or \emph{RUNNING} state.

\autoref{fig:methodology:job-task} illustrates the \ac{UML} class diagram for the job entity on the left side, lisitng all important attributes and methods of a job. The language attribute defines the programming langugae of the source code that has been compiled to a WebAssembly binary file.
\\~\\
Job entities are stored on the database and can be loaded from the backend. Additionally, the backend provides a compromised \ac{DTO} of each job object for workers or administrator users.
\begin{figure}[htbp]
    \centering
    \includegraphics[width=0.75\textwidth]{gfx/figures/Job-Task.png}
    \caption{\acs{UML} Class Diagram: Job \& Task}
    \label{fig:methodology:job-task}
\end{figure}

\subsubsection{Task}
Each job is divided into multiple tasks and ideally has each task an equal computational workload. These tasks are distributed to the participating worker nodes. Tasks are unique objects, each holding the specific input parameters that describe the particular portion of the job to which the task is assigned. \autoref{fig:methodology:job-task} displays the \ac{UML} class diagram for the task entity on the right side. The relationship between the job and task entities is defined as One-to-Many, meaning a job can consist of multiple tasks, but each task is always assigned to a single job.

A batch is defined to be a subset of tasks, which are all assigned to the same job. When the state of a job becomes \emph{ACTIVE} the backend enriches the \emph{tasks} attribute of this job object to match the current batch of this job.
\\~\\
Tasks are not stored in the database, but dynamically generated in the backend under utilization of the corresponding task input argument file stored on the web server. These tasks are then distributed to workers for computation.

\subsubsection{User}
A user represents a client that accesses the WebArgo platform through the frontend application. The actions a user can perform are determined by its assigned user role. This user role can either have the value \emph{User} for limited access or \emph{Admin} for full access to all features. \autoref{fig:methodology:client} displays the \ac{UML} class diagram for the user entity at the top of the illustration. This user information is stored in the database and used for the security measures, later described in \autoref{sec:implementation:authentication}. 

\begin{figure}[htbp]
    \centering
    \includegraphics[width=0.75\textwidth]{gfx/figures/Client.png}
    \caption{\acs{UML} Class Diagram: User | Worker \& Administrator}
    \label{fig:methodology:client}
\end{figure}
~\\
Workers receive tasks from the current batch of the active job, compute these tasks, and send the corresponding results to the backend. \autoref{fig:methodology:client} presents the \ac{UML} class diagram of the worker entity on the left side, listing all relevant attributes and methods. As workers connect to the frontend via web browsers, the browser's user agent is utilized to individually characterize each worker. This so-called browser user agent can be accessed inside the clients browser environment and provides information about the client's hardware resources and operating system. A worker object is dynamically generated in the frontend and backend application when a worker client accesses WebArgo.
\\~\\~\\
An administrator user can perform the same actions as a worker, but also has additional capabilities. Only administrators have the ability to set or change the state of jobs and to oversee the progress of all jobs. \autoref{fig:methodology:client} illustrates the \ac{UML} class diagram of the administrator entity on the right side. When an administrator accesses the WebArgo platform an corresponding administrator object is dynamically generated in the frontend and backend application.

\section{Scheduling}
\label{sec:implementation:scheduling}
The backend is responsible for distributing tasks from the current batch to participating workers. To maximize performance, the backend keeps track of scheduled tasks to prevent duplicate task distribution among workers. The scheduling of tasks follows the \ac{FIFO} methodology, scheduling the tasks in sequential order. As described in \autoref{sec:implementation:communication}, each job object has a timeout attribute for its tasks. Only if a task remains incomplete after its designated timeout period has expired, can this task be rescheduled. This mechanism serves to mitigate issues that arise from malfunctioning, disconnected, or straggling workers.
\\~\\
The scheduling mechanism could be enhanced through the implementation of performance-aware distribution, allocating computationally intensive tasks to workers identified to possess superior hardware. Furthermore, these identified stronger workers could execute multiple WebWorker with different tasks, performing parallel task execution on a single worker instance.

\section{Security through Authentication \& Authorization}
\label{sec:implementation:authentication}
To prevent malicious use, particularly in critical processes managed by administrators, the system implements a authentication mechanism. This is achieved through user credentials combined with a \ac{JWT}.

Authentication and Authorization are two key security concepts that work together to protect systems and data. Authentication is used to verify a clients identity - proving they are who they claim to be - in this platform realized through user credentials. Once a user is authenticated, Authorization determines what they're allowed to do within the system by checking their permissions and access rights. The \emph{userRole} attribute in the user object defines the specific access permissions. Together, these processes ensure that users are verified and these authenticated users can only access the resources and perform the actions appropriate for their role or permission level.

A \ac{JWT} implements a compact, \acs{URL}-safe methodology for secure information transmission between parties as a \ac{JSON} object \cite{implementation:jwt}. The token architecture comprises three dot-separated components: a header that describes the token type and signing algorithm, a payload containing the encoded data and a signature to verify the token's authenticity \cite{implementation:jwt}. \ac{JWT}s are commonly used in authentication and authorization processes, where information like user identity and permissions needs to be securely shared between services. The token can be verified by other services using the signature, eliminating the need to repeatedly validate credentials against a database. This makes \ac{JWT}s particularly useful in modern web applications and microservices architectures where secure, stateless authentication is required. \cite{implementation:jwt}

Upon successful user authentication, when a client connects to the platform, the backend generates and sings a \ac{JWT}, which is used to authenticate subsequent requests of this user. This generated \ac{JWT} is stored as a browser cookie on the client side which remains valid during a two-day expiration period. The data stored in the \ac{JWT} payload is a \ac{JSON} object containing the attributes \emph{id}, \emph{name}, and \emph{userRole} corresponding to the authenticated user. Clients transmit their specific \ac{JWT} with every \ac{API} request or when establishing a WebSocket connection. Hence, the backend can then use this token to authenticate the user and check if the user is authorized for this action.

\section{Backend}
\label{sec:implementation:backend}
This section describes the usage and features of the backend, which is implemented using the NestJS \cite{methodology:nestjs} framework. The backend application serves three primary functions: 
\begin{itemize}
    \item Data management for users, jobs, tasks, and their associated results.
    \item Handling the WebSockets communication with all connected clients.
    \item Task distribution to participating workers.
\end{itemize}
To manage and interact with jobs or users, the backend exposes multiple endpoints through a \acs{REST}ful \ac{API}. \autoref{fig:implementation:backend} presents a list of all available endpoints and their corresponding \acs{HTTP} request method. Due to their handling of critical and sensitive data, all endpoints are secured through \ac{JWT} authorization and always require administrator privileges, granted only for users with the \emph{Admin} \emph{userRole}. The login endpoint represents the only exception to this security measure, as it serves to generate a \ac{JWT} in the first place by validating a users credentials, and therefore operates without any \ac{JWT} authorization.

The backend \ac{API} implements comprehensive \ac{CRUD} operations for both job and user entities. These endpoints enable the retrieval of individual objects or complete object collections, the creation of new objects, and the modification or removal of existing objects. Furthermore, the \ac{API} exposes four additional endpoints to change the state of jobs and therefore providing the features to control job execution of WebArgo. Upon job activation, a batch associated to this job is generated and all connected workers are triggered to initiate the initialization of their corresponding WebAssembly environment. Starting a job initiates the process of distributing tasks from its batch to \emph{ready} participating workers. When a job is stopped, its current state is persisted on the web server as described in \autoref{sec:implementation:persistence}. The job reset operation permanently removes all progress associated with a job and generates a new unprocessed batch containing the initial tasks.
\begin{figure}[bth]
    \myfloatalign
    \subfloat[Job \ac{API} endpoints]{
    \label{fig:implementation:backend:job}
    \includegraphics[width=.25\linewidth]{gfx/figures/Job_API.png}
    } \quad \quad
    \subfloat[User \ac{API} endpoints]{
    \label{fig:implementation:backend:user}
    \includegraphics[width=.25\linewidth]{gfx/figures/User_API.png}
    }
    \caption{Available Backend \ac{API} endpoints}
    \label{fig:implementation:backend}
\end{figure}
~\\
Additionally, the backend \ac{API} serves static files of the compiled WebAssembly binaries and the optional required JavaScript glue code files for each job. These files are required to initializes a WebAssembly environment, and are therefore requested by worker nodes.

Furthermore, the backend implements the WebSocket-based communication protocol as described in \autoref{sec:implementation:communication} that manages bidirectional data streams between the backend and connected clients. This implementation enables performant task distribution and task result collection as well as real-time comunication for job progress monitoring while minimizing the overall communication overhead.

\section{Frontend}
\label{sec:implementation:frontend}
This section describes the features and main web pages served by the frontend. It is implemented utilizing the NextJS \cite{methodology:nextjs} framework in combination with styled components from the React Bootstrap library \cite{implementation:bootstrap}. The core features implemented in the frontend application are:
\begin{itemize}
    \item Providing an easy to use interface to interact with WebArgo
    \item Establishing a WebSocket connection between clients and the backend
    \item Computing tasks with WebAssembly in a WebWorker
    \item Serving a dashboard for administrators to manage jobs
\end{itemize}
The frontend implements two main web pages. These are the \emph{Dashboard Page} and the \emph{Client Page}.

\subsection{Dashboard Page}
\label{subsec:implementation:dashboard-page}
The \emph{Dashboard Page}, displayed in \autoref{fig:implementation:dashboard-page}, is only accessible for administrators. This page is used to monitor all connected workers and the progress of jobs in real-time. Furthermore, are administrators able to manage jobs through this web page.
\begin{figure}[htbp]
    \myfloatalign
    \subfloat[Dashboard Page: Job List]{
        \label{fig:implementation:job-list}
        \includegraphics[width=.8\textwidth]{gfx/figures/job-list.png}
    } \\
    \subfloat[Dashboard Page: Active Job]{
        \label{fig:implementation:active-job}
        \includegraphics[width=.8\linewidth]{gfx/figures/active-job-2.png}
    }
    \caption{Frontend Dashboard Page}
    \label{fig:implementation:dashboard-page}
\end{figure}
~\\
On the left side of the \emph{Dashboard Page} is a component located to manage the jobs of the platform. It holds a list of all existing jobs and their current progress, displayed in \autoref{fig:implementation:job-list}. Above this list of jobs appears a new component when a job is activated by an administrator.
~\\
\autoref{fig:implementation:active-job} displays a view of this active job component on the left side. This component visualizes the progress of an active job and displays the gathered results of complete tasks after each batch completion in real-time. Additionally it provides an interface to interact with this job. When a button of this interface is clicked, a corresponding \acs{HTTP} request, containing the administrator's \ac{JWT}, is transmitted to the backend. After all tasks are completed and the state of the job is \emph{DONE}, this component presents information about the total run time of the job. 

On the right side is another component located that displays a list of all connected workers, also in real-time. This list contains information about the initialization progress as well as the hardware and operating system for each worker with an active WebSocket connection to the backend. A view of this list is displayed in \autoref{fig:implementation:active-job} on the right side.

\subsection{Client Page}
\label{subsec:implementation:client-page}
The \emph{Client Page} represents the logic for workers and can be accessed by workers or administrators. When a client is accessing this web page the browser process becomes a participating worker in the WebArgo volunteer computing platform. First the worker establishes a WebSocket connection to the backend to receive job and task data and to send task results as described in \autoref{sec:implementation:communication}. If this connection is successful the browser user agent of the workers device is extracted and transmitted to the backend. When a job is active or running the connected worker is creating an independent browser thread - the WebWorker - and initializes the corresponding WebAssembly environment inside this WebWorker. Each task that the worker is receiving is forwarded for computation to this WebWorker. The performance and availability of the main browser thread, responsible for the \ac{UI} and WebSocket connection, is not affected by the processing of incoming tasks, since the intensive WebAssembly computation is handled inside the separate and independent WebWorker.

\autoref{fig:implementation:client-page} displays the view of the \emph{Client Page}. The main component of this page presents various information about the worker, like the status of its WebSocket connection, the current actively supported job and statistics about all tasks that have been completed by this worker. Therefore workers can always monitor the current status of their device, task computation and what kind of job they are supporting. Additionally, the usage of a WebWorker for the WebAssembly execution prevents the user page from becoming unresponsive or frozen during task computation. This enhances the user experience and also is supposed to prevent workers from potentially disconnecting because they could experience a unexpected frozen \ac{UI}.
\clearpage
\begin{figure}[htbp]
    \centering
    \includegraphics[width=0.8\textwidth]{gfx/figures/client-page.png}
    \caption{Frontend Client Page}
    \label{fig:implementation:client-page}
\end{figure}
~\\
In the current state of WebArgo the worker is implemented to initialize and execute WebAssembly binaries complied from either C \& C++, Go or Python source code. The support of further WebAssembly targeting languages can be added effortlessly to this work, since the usage of WebWorker scripts is handled generic. To support specific programming languages each WebWorker script requires additional unique glue code during the WebAssembly initialization process.

\section{Benchmark}
\label{sec:implementation:benchmark}
As described in \autoref{sec:methodology:benchmark}, the visualization of the Mandelbrot set is used to benchmark the performance of the volunteer computing platform in \autoref{ch:evaluation}. To achieve this, this job is distributed among multiple workers through the WebArgo platform and the resulting total execution time of this approach is then compared to the total execution time of the same job on a single device in a native environment.

Since the Mandelbrot set represents a subset of complex numbers, it is visualized in a two-dimensional coordinate system representing the two-dimensional complex plane. The primary region of interest is located in an area bounded by $1$ to $-2$ on the X-axis (real numbers) and $1.5$ to $-1.5$ on the Y-axis (imaginary numbers). This area is partitioned into 100 equally sized sections, forming a 10x10 grid. \autoref{fig:implementation:mandelbrot-page} visualizes the Mandelbrot set's primary region of interest divided into 100 equal areas.
\clearpage 
\begin{figure}[htbp]
    \centering
    \includegraphics[width=0.95\textwidth]{gfx/figures/mandelbrot-page.png}
    \caption{Frontend Mandelbrot Page}
    \label{fig:implementation:mandelbrot-page}
\end{figure}
~\\
Each grid tile represents the input arguments of a distinct task. An additional task computes the entire area of the Mandelbrot set at a lower resolution as a thumbnail, resulting in a total of 101 unique tasks for the benchmark job. Each of these tasks is generating a \acs{PNG} file that visualizes the corresponding Mandelbrot set tile, utilizing the same color scheme as illustrated in \autoref{fig:methodology:mandelbrot}. Each task computes the Mandelbrot condition for 2.25 million complex numbers $c$ within a 1500x1500 square region. Each of these 2.25 million $c$ represents a pixel in the \acs{PNG} file generated by the task. On task completion the generated \acs{PNG} file is transmitted to the backend through the WebSocket connection and stored on the web server in a corresponding directory. After the job is successfully completed an interactive visualization of the Mandelbrot set becomes accessible via the \emph{Mandelbrot page}, served by the frontend application. This specific page is displayed in \autoref{fig:implementation:mandelbrot-page}. Each tile and the thumbnail picture is computed and generated by a worker, representing the workload of a single task. By clicking on one of these tiles on the thumbnail, the corresponding \acs{PNG} picture can be viewed in its original resolution underneath the interactive Mandelbrot visualization component.

The WebAssembly-compatible Go source code, which is executed by workers during Task computation, is presented in \autoref{app:code:mandelbrot2}. The corresponding native Go implementation, executed in comparision on a single device, can be found in \autoref{app:code:mandelbrot3}.

\section{Challenges}
\label{sec:implementation:challenges}
This section lists and describes the  challenges that occurred during the development process of the volunteer computing platform.
\newline
\newline
\textbf{Generic implementation of Jobs:}
\newline
The platform is designed to allow the execution of any kind of custom job. To ensure a flexible development of new jobs, the job entity and the handling of tasks, WebAssembly binaries, and input arguments are implemented using generic patterns throughout the architecture in all components. This enables the distribution of jobs across the platform, written in any supported programming language and with any amount of tasks, effortlessly with a minimal programming overhead.
\newline
\newline
\textbf{Loading glue code and WebAssembly files from external source (Backend) inside WebWorker:}
\newline
To maintain simplicity, all WebAssembly binaries and their associated glue code files are centrally stored in their corresponding job directory on the web server, hence ensuring that all files specific to a job remain in a single location. However, the initialization of the WebAssembly environment in the \emph{Client Page} of the frontend requires fetching and incorporating files from the backend as a third-party source in this design. This behavior presented a challenge during the implementation, as the glue code scripts expect its corresponding WebAssembly binary to be located in the same local directory instead of being externaly loaded. It was not trivial to reproduce this behavior inside the WebWorker.
\newline
\newline
\textbf{Handling the Input and Output of WebAssembly code:}
\newline
A string array format, similar to conventional command-line input arguments, has been selected as the uniform input type across all tasks. This standardized format enables simple parsing of arguments from the input text files and can also be effortlessly forwarded to WebAssembly functions within the JavaScript runtime environment.

The implementation of a generic output format across all tasks and programming languages presented significant challenges. Especially for C and C++ source code, where the \emph{main} function is constrained to return always a single numeric value (\emph{int}). However, the developed solution supports the output of any primitive datatypes, lists of primitive datatypes, objects and also binaries, even for C and C++ applications. This behaviour applies to all kinds of tasks throughout all supported programming languages.

To enable binaries as task output the WebSocket connection between workers and the backend had to be adjusted. The default message size limit (\emph{maxHttpBufferSize}) of the Socket.IO \cite{methodology:websockets2} library has been raised to $100MB$ per message to ensure the transmission of larger files.
\newline
\newline
\textbf{Implement system recovery measures:}
\newline
The design and implementation of system recovery mechanisms, as described in \autoref{sec:implementation:persistence}, has represented a complex enhancement to the platform.
\newline
\newline
\textbf{Prevent duplicate Task execution \& ensure Job completion in case of malfunctioning Workers:}
\newline
The timeout mechanism described in \autoref{sec:implementation:scheduling}, comparable but not identical to the timeout implementation of XtremWeb \cite{relatedwork:xtremweb}, was a crucial enhancement that had to be integrated into the scheduling process. This timeout mechanism is used to prevent duplicate task execution and ensures rescheduling of aborted tasks.